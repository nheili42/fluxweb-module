% Options for packages loaded elsewhere
\PassOptionsToPackage{unicode}{hyperref}
\PassOptionsToPackage{hyphens}{url}
%
\documentclass[
]{article}
\usepackage{amsmath,amssymb}
\usepackage{iftex}
\ifPDFTeX
  \usepackage[T1]{fontenc}
  \usepackage[utf8]{inputenc}
  \usepackage{textcomp} % provide euro and other symbols
\else % if luatex or xetex
  \usepackage{unicode-math} % this also loads fontspec
  \defaultfontfeatures{Scale=MatchLowercase}
  \defaultfontfeatures[\rmfamily]{Ligatures=TeX,Scale=1}
\fi
\usepackage{lmodern}
\ifPDFTeX\else
  % xetex/luatex font selection
\fi
% Use upquote if available, for straight quotes in verbatim environments
\IfFileExists{upquote.sty}{\usepackage{upquote}}{}
\IfFileExists{microtype.sty}{% use microtype if available
  \usepackage[]{microtype}
  \UseMicrotypeSet[protrusion]{basicmath} % disable protrusion for tt fonts
}{}
\makeatletter
\@ifundefined{KOMAClassName}{% if non-KOMA class
  \IfFileExists{parskip.sty}{%
    \usepackage{parskip}
  }{% else
    \setlength{\parindent}{0pt}
    \setlength{\parskip}{6pt plus 2pt minus 1pt}}
}{% if KOMA class
  \KOMAoptions{parskip=half}}
\makeatother
\usepackage{xcolor}
\usepackage[margin=1in]{geometry}
\usepackage{color}
\usepackage{fancyvrb}
\newcommand{\VerbBar}{|}
\newcommand{\VERB}{\Verb[commandchars=\\\{\}]}
\DefineVerbatimEnvironment{Highlighting}{Verbatim}{commandchars=\\\{\}}
% Add ',fontsize=\small' for more characters per line
\usepackage{framed}
\definecolor{shadecolor}{RGB}{248,248,248}
\newenvironment{Shaded}{\begin{snugshade}}{\end{snugshade}}
\newcommand{\AlertTok}[1]{\textcolor[rgb]{0.94,0.16,0.16}{#1}}
\newcommand{\AnnotationTok}[1]{\textcolor[rgb]{0.56,0.35,0.01}{\textbf{\textit{#1}}}}
\newcommand{\AttributeTok}[1]{\textcolor[rgb]{0.13,0.29,0.53}{#1}}
\newcommand{\BaseNTok}[1]{\textcolor[rgb]{0.00,0.00,0.81}{#1}}
\newcommand{\BuiltInTok}[1]{#1}
\newcommand{\CharTok}[1]{\textcolor[rgb]{0.31,0.60,0.02}{#1}}
\newcommand{\CommentTok}[1]{\textcolor[rgb]{0.56,0.35,0.01}{\textit{#1}}}
\newcommand{\CommentVarTok}[1]{\textcolor[rgb]{0.56,0.35,0.01}{\textbf{\textit{#1}}}}
\newcommand{\ConstantTok}[1]{\textcolor[rgb]{0.56,0.35,0.01}{#1}}
\newcommand{\ControlFlowTok}[1]{\textcolor[rgb]{0.13,0.29,0.53}{\textbf{#1}}}
\newcommand{\DataTypeTok}[1]{\textcolor[rgb]{0.13,0.29,0.53}{#1}}
\newcommand{\DecValTok}[1]{\textcolor[rgb]{0.00,0.00,0.81}{#1}}
\newcommand{\DocumentationTok}[1]{\textcolor[rgb]{0.56,0.35,0.01}{\textbf{\textit{#1}}}}
\newcommand{\ErrorTok}[1]{\textcolor[rgb]{0.64,0.00,0.00}{\textbf{#1}}}
\newcommand{\ExtensionTok}[1]{#1}
\newcommand{\FloatTok}[1]{\textcolor[rgb]{0.00,0.00,0.81}{#1}}
\newcommand{\FunctionTok}[1]{\textcolor[rgb]{0.13,0.29,0.53}{\textbf{#1}}}
\newcommand{\ImportTok}[1]{#1}
\newcommand{\InformationTok}[1]{\textcolor[rgb]{0.56,0.35,0.01}{\textbf{\textit{#1}}}}
\newcommand{\KeywordTok}[1]{\textcolor[rgb]{0.13,0.29,0.53}{\textbf{#1}}}
\newcommand{\NormalTok}[1]{#1}
\newcommand{\OperatorTok}[1]{\textcolor[rgb]{0.81,0.36,0.00}{\textbf{#1}}}
\newcommand{\OtherTok}[1]{\textcolor[rgb]{0.56,0.35,0.01}{#1}}
\newcommand{\PreprocessorTok}[1]{\textcolor[rgb]{0.56,0.35,0.01}{\textit{#1}}}
\newcommand{\RegionMarkerTok}[1]{#1}
\newcommand{\SpecialCharTok}[1]{\textcolor[rgb]{0.81,0.36,0.00}{\textbf{#1}}}
\newcommand{\SpecialStringTok}[1]{\textcolor[rgb]{0.31,0.60,0.02}{#1}}
\newcommand{\StringTok}[1]{\textcolor[rgb]{0.31,0.60,0.02}{#1}}
\newcommand{\VariableTok}[1]{\textcolor[rgb]{0.00,0.00,0.00}{#1}}
\newcommand{\VerbatimStringTok}[1]{\textcolor[rgb]{0.31,0.60,0.02}{#1}}
\newcommand{\WarningTok}[1]{\textcolor[rgb]{0.56,0.35,0.01}{\textbf{\textit{#1}}}}
\usepackage{graphicx}
\makeatletter
\def\maxwidth{\ifdim\Gin@nat@width>\linewidth\linewidth\else\Gin@nat@width\fi}
\def\maxheight{\ifdim\Gin@nat@height>\textheight\textheight\else\Gin@nat@height\fi}
\makeatother
% Scale images if necessary, so that they will not overflow the page
% margins by default, and it is still possible to overwrite the defaults
% using explicit options in \includegraphics[width, height, ...]{}
\setkeys{Gin}{width=\maxwidth,height=\maxheight,keepaspectratio}
% Set default figure placement to htbp
\makeatletter
\def\fps@figure{htbp}
\makeatother
\setlength{\emergencystretch}{3em} % prevent overfull lines
\providecommand{\tightlist}{%
  \setlength{\itemsep}{0pt}\setlength{\parskip}{0pt}}
\setcounter{secnumdepth}{-\maxdimen} % remove section numbering
\ifLuaTeX
  \usepackage{selnolig}  % disable illegal ligatures
\fi
\usepackage{bookmark}
\IfFileExists{xurl.sty}{\usepackage{xurl}}{} % add URL line breaks if available
\urlstyle{same}
\hypersetup{
  pdftitle={Flux web module},
  pdfauthor={Nate Heili},
  hidelinks,
  pdfcreator={LaTeX via pandoc}}

\title{Flux web module}
\author{Nate Heili}
\date{2025-03-27}

\begin{document}
\maketitle

\section{Constructing energetic food webs using
fluxweb}\label{constructing-energetic-food-webs-using-fluxweb}

Here, I explore the math and theory behind \emph{fluxweb} (Gauzens et
al.~2018), an R package that estimates energy fluxes between consumers
and resources in food webs.

\begin{Shaded}
\begin{Highlighting}[]
\CommentTok{\#install.packages("fluxweb")}
\FunctionTok{library}\NormalTok{(fluxweb)}
\end{Highlighting}
\end{Shaded}

\begin{verbatim}
## Warning: package 'fluxweb' was built under R version 4.4.3
\end{verbatim}

\begin{Shaded}
\begin{Highlighting}[]
\FunctionTok{library}\NormalTok{(tidyverse)}
\end{Highlighting}
\end{Shaded}

\begin{verbatim}
## -- Attaching core tidyverse packages ------------------------ tidyverse 2.0.0 --
## v dplyr     1.1.4     v readr     2.1.5
## v forcats   1.0.0     v stringr   1.5.1
## v ggplot2   3.5.1     v tibble    3.2.1
## v lubridate 1.9.4     v tidyr     1.3.1
## v purrr     1.0.2     
## -- Conflicts ------------------------------------------ tidyverse_conflicts() --
## x dplyr::filter() masks stats::filter()
## x dplyr::lag()    masks stats::lag()
## i Use the conflicted package (<http://conflicted.r-lib.org/>) to force all conflicts to become errors
\end{verbatim}

\begin{Shaded}
\begin{Highlighting}[]
\NormalTok{??fluxweb }\CommentTok{\# explore functions and vignettes from the package}
\end{Highlighting}
\end{Shaded}

\begin{verbatim}
## starting httpd help server ... done
\end{verbatim}

\subsection{Data structure and
requirements}\label{data-structure-and-requirements}

Below, I load the package vignette dataset ``species.level''. This
dataset contains the matrix describing trophic interactions from a soil
food web (Digel et al.~2014, Oikos) as well as ecological information on
the species: biomasses, body masses, assimilation efficiencies, and
species names.

\begin{Shaded}
\begin{Highlighting}[]
\CommentTok{\# Upload the species level data provided in the package vignette}
\FunctionTok{data}\NormalTok{(}\StringTok{"species.level"}\NormalTok{)}

\CommentTok{\# Explore the structure of the data}
\FunctionTok{str}\NormalTok{(species.level)}
\end{Highlighting}
\end{Shaded}

\begin{verbatim}
## List of 5
##  $ mat         : num [1:62, 1:62] 0 0 0 0 0 0 0 0 0 0 ...
##  $ biomasses   : num [1:62] 114.5 436.6 1 8.46 34.06 ...
##  $ bodymasses  : num [1:62] 0.05 9.096 1 0.132 2.129 ...
##  $ efficiencies: num [1:62] 0.906 0.906 0.545 0.906 0.906 0.906 0.906 0.906 0.906 0.906 ...
##  $ names       : 'noquote' chr [1:62] "Achipteria coleoptrata" "Agriotes aterrimus (juv)" "algae" "Atheta sp. (juv)" ...
\end{verbatim}

\subsection{Calculate metabolic rates}\label{calculate-metabolic-rates}

The metabolic rates represent energy losses per unit biomass for each
species, based on allometric relationships (how metabolism scales with
body size). This equation comes from the metabolic theory of ecology
(Brown et al.~2004), where metabolic rate scales with body mass
following a power law:

\[
B = B_0 M^b
\]

where: \(B\) is the metabolic rate, \(B_0\) is a normalization constant,
\(M\) is body mass, and \(b\) is the scaling exponent (often around 0.75
for metabolic rates in animals).

There has been numerous studies that argue the nuance of these parameter
values, and if available, the user can customize them by organism type
or group (e.g.~ectotherm, endotherm, invertebrate, etc.). The loaded
dataset in this example does not come with organism group so we won't
worry about that for now. The average values from the MTE over all
species groups for parameters a and b are 0.71 and −0.25. Thus:

\begin{Shaded}
\begin{Highlighting}[]
\CommentTok{\# Calculate metabolic losses}
\NormalTok{losses }\OtherTok{=} \FloatTok{0.71} \SpecialCharTok{*}\NormalTok{ species.level}\SpecialCharTok{$}\NormalTok{bodymasses}\SpecialCharTok{\^{}}\NormalTok{(}\SpecialCharTok{{-}}\FloatTok{0.25}\NormalTok{)}
\end{Highlighting}
\end{Shaded}

\subsection{Feeding efficiencies}\label{feeding-efficiencies}

The final parameter needed to calculate energy fluxes is a vector of
feeding efficiencies. Because species' physiological losses were
estimated using metabolic rates, assimilation efficiency should be used.
Assimilation efficiency (AE) represents the proportion of consumed food
that an organism can actually use for metabolism and growth. Not all
ingested food is useful---some is lost as waste (e.g., undigested
material in feces). Additonally, because not all food is nutritionally
the same, AE can be customized based on the type of prey eaten. In this
dataset, AE values were predefined for different prey types
(species.level\$efficiencies), where 0.906 is animal, 0.545 is plant,
and 0.158 is detritus.

\section{Calculate energy fluxes}\label{calculate-energy-fluxes}

Now that we have our ducks in a row, we use the \emph{fluxing} function
to calculate energy fluxes between species. This function solves for the
equilibrium fluxes needed to maintain biomass balance in the system. For
each species, energy gains (incoming fluxes * efficiency) must balance
or equal energy losses (metabolism + outgoing fluxes).

\begin{Shaded}
\begin{Highlighting}[]
\NormalTok{mat.fluxes }\OtherTok{\textless{}{-}} \FunctionTok{fluxing}\NormalTok{(species.level}\SpecialCharTok{$}\NormalTok{mat,}
\NormalTok{                      species.level}\SpecialCharTok{$}\NormalTok{biomasses,}
\NormalTok{                      losses,}
\NormalTok{                      species.level}\SpecialCharTok{$}\NormalTok{efficiencies)}
\end{Highlighting}
\end{Shaded}

This step is where the package provides user flexibility, and it is
important to note the default behavior of the of the \emph{fluxing}
function:\\
Note, there is much more detail in the journal article!!

\subsubsection{Influence of prey availability on feeding
(bioms.prefs)}\label{influence-of-prey-availability-on-feeding-bioms.prefs}

The default behavior of \emph{fluxing} assumes that predators adjust
their feeding preferences based on the biomass of available prey. This
means that when bioms.pref = TRUE, the function scales the diet
preferences from the food web matrix according to prey biomasses. As a
result, more abundant prey are consumed at higher proportions, while
less abundant prey contribute less to the total energy flux.

However, users have the flexibility to override this default behavior.
Setting bioms.pref = FALSE keeps the raw diet preferences unchanged,
meaning that predators will consume prey in fixed proportions regardless
of their biomass. This may be useful in cases where feeding behavior is
constrained by factors other than prey abundance, such as specialist
diets or strong prey selection mechanisms.

\subsubsection{Influence of metabolic losses
(bioms.losses)}\label{influence-of-metabolic-losses-bioms.losses}

The default behavior of \emph{fluxing} assumes metabolic losses are the
product of species' biomasses (species.level\$biomasses) and a loss rate
per unit biomass (the losses vector calculated above using parameters
from the MTE). This approach assumes that metabolic losses scale with
body size, and larger species will have greater total metabolic losses.

However, if direct respiration measurements are available, users can
override this scaling setting. In this case, the model uses observed
respiration rates or other population-level metabolic data instead of
relying on body mass scaling.

\subsubsection{Influence of assimilation efficiency defined by prey
(ef.level)}\label{influence-of-assimilation-efficiency-defined-by-prey-ef.level}

The defualt behavior ef.level = ``prey'' assumes that the assimilation
efficiencies are defined according to prey quality (resource defined).

However, if available, the user can define predator specific
assimilation efficiencies ef.level = ``pred'' (consumer defined).

\section{Visualize the energy flux
distributions}\label{visualize-the-energy-flux-distributions}

\begin{Shaded}
\begin{Highlighting}[]
\CommentTok{\# Convert the matrix to a vector of values for distribution analysis}
\NormalTok{flux\_values }\OtherTok{\textless{}{-}} \FunctionTok{as.vector}\NormalTok{(mat.fluxes)}
\NormalTok{flux\_values }\OtherTok{\textless{}{-}}\NormalTok{ flux\_values[flux\_values }\SpecialCharTok{\textgreater{}} \DecValTok{0}\NormalTok{]  }\CommentTok{\# Remove zero values (non{-}interactions)}

\CommentTok{\# Basic statistics}
\FunctionTok{summary}\NormalTok{(flux\_values)}
\end{Highlighting}
\end{Shaded}

\begin{verbatim}
##     Min.  1st Qu.   Median     Mean  3rd Qu.     Max. 
##    0.001    0.421    1.720   41.477    6.839 3220.344
\end{verbatim}

\begin{Shaded}
\begin{Highlighting}[]
\CommentTok{\# Histogram of flux values}
\FunctionTok{hist}\NormalTok{(flux\_values, }
     \AttributeTok{main =} \StringTok{"Distribution of Energy Fluxes"}\NormalTok{, }
     \AttributeTok{xlab =} \StringTok{"Energy flux (Joules/year)"}\NormalTok{,}
     \AttributeTok{col =} \StringTok{"skyblue"}\NormalTok{,}
     \AttributeTok{breaks =} \DecValTok{30}\NormalTok{)}
\end{Highlighting}
\end{Shaded}

\includegraphics{flux_web_vignette_files/figure-latex/unnamed-chunk-5-1.pdf}

\begin{Shaded}
\begin{Highlighting}[]
\CommentTok{\# Log{-}transformed histogram (often useful for flux data which can be skewed)}
\FunctionTok{hist}\NormalTok{(}\FunctionTok{log10}\NormalTok{(flux\_values), }
     \AttributeTok{main =} \StringTok{"Distribution of Energy Fluxes"}\NormalTok{, }
     \AttributeTok{xlab =} \StringTok{"ln(Energy flux [Joules/year])"}\NormalTok{,}
     \AttributeTok{col =} \StringTok{"lightgreen"}\NormalTok{,}
     \AttributeTok{breaks =} \DecValTok{30}\NormalTok{)}
\end{Highlighting}
\end{Shaded}

\includegraphics{flux_web_vignette_files/figure-latex/unnamed-chunk-5-2.pdf}

\begin{Shaded}
\begin{Highlighting}[]
\CommentTok{\# Density plot}
\FunctionTok{plot}\NormalTok{(}\FunctionTok{density}\NormalTok{(flux\_values), }
     \AttributeTok{main =} \StringTok{"Density Plot of Energy Fluxes"}\NormalTok{,}
     \AttributeTok{xlab =} \StringTok{"Energy Flux Value"}\NormalTok{)}
\end{Highlighting}
\end{Shaded}

\includegraphics{flux_web_vignette_files/figure-latex/unnamed-chunk-5-3.pdf}

\section{Incorporating environmental temperature into metaboloic
lossess}\label{incorporating-environmental-temperature-into-metaboloic-lossess}

Metabolic rate not only scales with body mass but also responds strongly
to environmental temperature, particularly in ectothermic organisms. To
reflect this, we can integrate temperature into metabolic loss estimates
following the approach of Schwarz et al.~(2017), which builds on the
metabolic theory of ecology (Brown et al.~2004).

\[
B = B_0 M^b e^{-\frac{E}{k_B T}}
\]

Where:\\
- \(B\) is the temperature-dependent metabolic rate,\\
- \(B_0\) is a normalization constant,\\
- \(M\) is fresh body mass (g),\\
- \(b\) is the allometric exponent (typically 0.75),\\
- \(E\) is the activation energy (eV),\\
- \(k_B\) is the Boltzmann constant (8.617 × 10⁻⁵ eV/K),\\
- \(T\) is absolute temperature in Kelvin (K).

Below, I will use this equation to simulate metabolic energy losses
across the community under two temperature scenarios, 4 and 20 degree C.

\begin{Shaded}
\begin{Highlighting}[]
\CommentTok{\#Constants from literature}
\NormalTok{k\_B }\OtherTok{\textless{}{-}} \FloatTok{8.617e{-}5}              \CommentTok{\# Boltzmann constant in eV/K}
\NormalTok{B0 }\OtherTok{\textless{}{-}} \FloatTok{0.88}                   \CommentTok{\# Normalization constant}
\NormalTok{b }\OtherTok{\textless{}{-}} \FloatTok{0.75}                    \CommentTok{\# Allometric exponent}
\NormalTok{E }\OtherTok{\textless{}{-}} \FloatTok{0.63}                    \CommentTok{\# Activation energy in eV}

\CommentTok{\#Losses at 4 degree C }
\NormalTok{losses\_4C }\OtherTok{\textless{}{-}}\NormalTok{ B0 }\SpecialCharTok{*}\NormalTok{ species.level}\SpecialCharTok{$}\NormalTok{bodymass}\SpecialCharTok{\^{}}\NormalTok{b }\SpecialCharTok{*} \FunctionTok{exp}\NormalTok{(}\SpecialCharTok{{-}}\NormalTok{E }\SpecialCharTok{/}\NormalTok{ (k\_B }\SpecialCharTok{*}\NormalTok{ (}\DecValTok{4} \SpecialCharTok{+} \FloatTok{273.15}\NormalTok{))) }\CommentTok{\#convert C to Kelvin}

\CommentTok{\#Losses at 20 degree C}
\NormalTok{losses\_20C }\OtherTok{\textless{}{-}}\NormalTok{ B0 }\SpecialCharTok{*}\NormalTok{ species.level}\SpecialCharTok{$}\NormalTok{bodymass}\SpecialCharTok{\^{}}\NormalTok{b }\SpecialCharTok{*} \FunctionTok{exp}\NormalTok{(}\SpecialCharTok{{-}}\NormalTok{E }\SpecialCharTok{/}\NormalTok{ (k\_B }\SpecialCharTok{*}\NormalTok{ (}\DecValTok{20} \SpecialCharTok{+} \FloatTok{273.15}\NormalTok{)))}
\end{Highlighting}
\end{Shaded}

\section{Calculate energy fluxes for cold and
warm}\label{calculate-energy-fluxes-for-cold-and-warm}

Now we do the same as above to estimate energy fluxes between species
for each of the temperature scenarios.

\begin{Shaded}
\begin{Highlighting}[]
\NormalTok{cold.fluxes }\OtherTok{\textless{}{-}} \FunctionTok{fluxing}\NormalTok{(species.level}\SpecialCharTok{$}\NormalTok{mat,}
\NormalTok{                      species.level}\SpecialCharTok{$}\NormalTok{biomasses,}
\NormalTok{                      losses\_4C,}
\NormalTok{                      species.level}\SpecialCharTok{$}\NormalTok{efficiencies)}

\NormalTok{warm.fluxes }\OtherTok{\textless{}{-}} \FunctionTok{fluxing}\NormalTok{(species.level}\SpecialCharTok{$}\NormalTok{mat,}
\NormalTok{                      species.level}\SpecialCharTok{$}\NormalTok{biomasses,}
\NormalTok{                      losses\_20C,}
\NormalTok{                      species.level}\SpecialCharTok{$}\NormalTok{efficiencies)}
\end{Highlighting}
\end{Shaded}

\subsection{Visualize energy flux distribtuions with
temperature-dependent metabolic
losses}\label{visualize-energy-flux-distribtuions-with-temperature-dependent-metabolic-losses}

\begin{Shaded}
\begin{Highlighting}[]
\CommentTok{\# Convert the matrix to a vector of values for distribution analysis}
\NormalTok{warm.flux.values }\OtherTok{\textless{}{-}} \FunctionTok{as.vector}\NormalTok{(warm.fluxes)}
\NormalTok{warm.flux.values }\OtherTok{\textless{}{-}}\NormalTok{ warm.flux.values[warm.flux.values }\SpecialCharTok{\textgreater{}}\DecValTok{0}\NormalTok{]}

\CommentTok{\# Convert the matrix to a vector of values for distribution analysis}
\NormalTok{cold.flux.values }\OtherTok{\textless{}{-}} \FunctionTok{as.vector}\NormalTok{(cold.fluxes)}
\NormalTok{cold.flux.values }\OtherTok{\textless{}{-}}\NormalTok{ cold.flux.values[cold.flux.values }\SpecialCharTok{\textgreater{}}\DecValTok{0}\NormalTok{]}

\CommentTok{\# Basic statistics}
\FunctionTok{summary}\NormalTok{(warm.flux.values)}
\end{Highlighting}
\end{Shaded}

\begin{verbatim}
##      Min.   1st Qu.    Median      Mean   3rd Qu.      Max. 
## 0.000e+00 6.200e-10 2.530e-09 1.515e-07 1.030e-08 5.946e-05
\end{verbatim}

\begin{Shaded}
\begin{Highlighting}[]
\FunctionTok{summary}\NormalTok{(cold.flux.values)}
\end{Highlighting}
\end{Shaded}

\begin{verbatim}
##      Min.   1st Qu.    Median      Mean   3rd Qu.      Max. 
## 0.000e+00 1.470e-10 6.010e-10 3.590e-08 2.440e-09 1.409e-05
\end{verbatim}

\begin{Shaded}
\begin{Highlighting}[]
\CommentTok{\# Combine your data into one data frame}
\NormalTok{flux.data }\OtherTok{\textless{}{-}} \FunctionTok{data.frame}\NormalTok{(}
  \AttributeTok{value =} \FunctionTok{c}\NormalTok{(cold.flux.values, warm.flux.values),}
  \AttributeTok{temperature =} \FunctionTok{factor}\NormalTok{(}\FunctionTok{rep}\NormalTok{(}\FunctionTok{c}\NormalTok{(}\StringTok{"Cold (4°C)"}\NormalTok{, }\StringTok{"Warm (20°C)"}\NormalTok{), }
                           \AttributeTok{times =} \FunctionTok{c}\NormalTok{(}\FunctionTok{length}\NormalTok{(cold.flux.values), }\FunctionTok{length}\NormalTok{(warm.flux.values))))}
\NormalTok{)}

\CommentTok{\# Plot histogram, another way to visualize the data}
\FunctionTok{ggplot}\NormalTok{(flux.data, }\FunctionTok{aes}\NormalTok{(}\AttributeTok{x =}\NormalTok{ value, }\AttributeTok{fill =}\NormalTok{ temperature)) }\SpecialCharTok{+}
  \FunctionTok{geom\_histogram}\NormalTok{(}\AttributeTok{position =} \StringTok{"identity"}\NormalTok{, }\AttributeTok{bins =} \DecValTok{50}\NormalTok{, }\AttributeTok{alpha =} \FloatTok{0.6}\NormalTok{) }\SpecialCharTok{+}
  \FunctionTok{scale\_x\_log10}\NormalTok{() }\SpecialCharTok{+} 
  \FunctionTok{scale\_fill\_manual}\NormalTok{(}
    \AttributeTok{values =} \FunctionTok{c}\NormalTok{(}\StringTok{"Cold (4°C)"} \OtherTok{=} \StringTok{"\#1f77b4"}\NormalTok{,  }
               \StringTok{"Warm (20°C)"} \OtherTok{=} \StringTok{"\#ff7f0e"}\NormalTok{)}
\NormalTok{  ) }\SpecialCharTok{+}
  \FunctionTok{labs}\NormalTok{(}
    \AttributeTok{x =} \StringTok{"Log Energy flux (J/s)"}\NormalTok{,}
    \AttributeTok{y =} \StringTok{"Count"}\NormalTok{,}
    \AttributeTok{fill =} \StringTok{"Temperature"}
\NormalTok{  ) }\SpecialCharTok{+}
  \FunctionTok{theme\_light}\NormalTok{() }\SpecialCharTok{+}
  \FunctionTok{theme}\NormalTok{(}\AttributeTok{text =} \FunctionTok{element\_text}\NormalTok{(}\AttributeTok{size =} \DecValTok{16}\NormalTok{))}
\end{Highlighting}
\end{Shaded}

\includegraphics{flux_web_vignette_files/figure-latex/unnamed-chunk-8-1.pdf}

\begin{Shaded}
\begin{Highlighting}[]
\CommentTok{\# Total energy flux in each temperature scenario}
\NormalTok{total.warm.flux }\OtherTok{\textless{}{-}} \FunctionTok{sum}\NormalTok{(warm.flux.values, }\AttributeTok{na.rm =} \ConstantTok{TRUE}\NormalTok{)}
\NormalTok{total.cold.flux }\OtherTok{\textless{}{-}} \FunctionTok{sum}\NormalTok{(cold.flux.values, }\AttributeTok{na.rm =} \ConstantTok{TRUE}\NormalTok{)}

\CommentTok{\# Create a data frame for plotting}
\NormalTok{total.flux.df }\OtherTok{\textless{}{-}} \FunctionTok{data.frame}\NormalTok{(}
  \AttributeTok{temperature =} \FunctionTok{c}\NormalTok{(}\StringTok{"Cold (4°C)"}\NormalTok{, }\StringTok{"Warm (20°C)"}\NormalTok{),}
  \AttributeTok{total\_flux =} \FunctionTok{c}\NormalTok{(total.cold.flux, total.warm.flux)}
\NormalTok{)}

\CommentTok{\# Bar plot}
\FunctionTok{ggplot}\NormalTok{(total.flux.df, }\FunctionTok{aes}\NormalTok{(}\AttributeTok{x =}\NormalTok{ temperature, }\AttributeTok{y =}\NormalTok{ total\_flux, }\AttributeTok{fill =}\NormalTok{ temperature)) }\SpecialCharTok{+}
  \FunctionTok{geom\_col}\NormalTok{(}\AttributeTok{width =} \FloatTok{0.6}\NormalTok{) }\SpecialCharTok{+}
  \FunctionTok{scale\_fill\_manual}\NormalTok{(}\AttributeTok{values =} \FunctionTok{c}\NormalTok{(}\StringTok{"Cold (4°C)"} \OtherTok{=} \StringTok{"\#1f77b4"}\NormalTok{, }\StringTok{"Warm (20°C)"} \OtherTok{=} \StringTok{"\#ff7f0e"}\NormalTok{)) }\SpecialCharTok{+}
  \FunctionTok{labs}\NormalTok{(}
    \AttributeTok{title =} \StringTok{"Total Energy Flux in Food Web"}\NormalTok{,}
    \AttributeTok{y =} \StringTok{"Total Flux (J/s)"}\NormalTok{,}
    \AttributeTok{x =} \StringTok{"Temperature"}\NormalTok{,}
    \AttributeTok{fill =} \StringTok{""}
\NormalTok{  ) }\SpecialCharTok{+}
  \FunctionTok{theme\_light}\NormalTok{()}
\end{Highlighting}
\end{Shaded}

\includegraphics{flux_web_vignette_files/figure-latex/unnamed-chunk-8-2.pdf}

Although the structure of the distribution stayed the same, warming
shifts the distribution toward higher flux values, driving greater total
energy flux in the food web. Therefore, if community richness and
biomass stay the same (as in this scenario), organisms in the warmed
system must increase the rate at which they consumer energy. This
results in a faster flow of energy through trophic interactions and an
overall increase in energy turnover within the food web. Alternatively,
organisms may not be able to keep up with metabolic demands and be lost
or reduced.

\#Removing vulnerable species from the food web Warming might remove
vulnerable species (in this case, those with low biomasses). To model
this, remove all species in the matrix with log(biomass) \textless{} 1.
Then calculate the new energy flux distribution and compare to the cold
scenario.

\begin{Shaded}
\begin{Highlighting}[]
\CommentTok{\# Identify species to keep (biomasses ≤ log(1) = 0)}
\NormalTok{keep }\OtherTok{\textless{}{-}} \FunctionTok{log}\NormalTok{(species.level}\SpecialCharTok{$}\NormalTok{biomasses) }\SpecialCharTok{\textgreater{}=} \DecValTok{1}

\CommentTok{\# Subset each component of the list accordingly}
\NormalTok{warm.species.level }\OtherTok{\textless{}{-}} \FunctionTok{list}\NormalTok{(}
  \AttributeTok{mat =}\NormalTok{ species.level}\SpecialCharTok{$}\NormalTok{mat[keep, keep],  }\CommentTok{\# subset rows and columns}
  \AttributeTok{biomasses =}\NormalTok{ species.level}\SpecialCharTok{$}\NormalTok{biomasses[keep],}
  \AttributeTok{bodymasses =}\NormalTok{ species.level}\SpecialCharTok{$}\NormalTok{bodymasses[keep],}
  \AttributeTok{efficiencies =}\NormalTok{ species.level}\SpecialCharTok{$}\NormalTok{efficiencies[keep],}
  \AttributeTok{names =}\NormalTok{ species.level}\SpecialCharTok{$}\NormalTok{names[keep]}
\NormalTok{)}

\NormalTok{losses\_20C}\FloatTok{.2} \OtherTok{\textless{}{-}}\NormalTok{ B0 }\SpecialCharTok{*}\NormalTok{ warm.species.level}\SpecialCharTok{$}\NormalTok{bodymass}\SpecialCharTok{\^{}}\NormalTok{b }\SpecialCharTok{*} \FunctionTok{exp}\NormalTok{(}\SpecialCharTok{{-}}\NormalTok{E }\SpecialCharTok{/}\NormalTok{ (k\_B }\SpecialCharTok{*}\NormalTok{ (}\DecValTok{20} \SpecialCharTok{+} \FloatTok{273.15}\NormalTok{)))}


\NormalTok{warm.fluxes}\FloatTok{.2} \OtherTok{\textless{}{-}} \FunctionTok{fluxing}\NormalTok{(warm.species.level}\SpecialCharTok{$}\NormalTok{mat,}
\NormalTok{                      warm.species.level}\SpecialCharTok{$}\NormalTok{biomasses,}
\NormalTok{                      losses\_20C}\FloatTok{.2}\NormalTok{,}
\NormalTok{                      warm.species.level}\SpecialCharTok{$}\NormalTok{efficiencies)}


\CommentTok{\# Convert the matrix to a vector of values for distribution analysis}
\NormalTok{warm.flux.values }\OtherTok{\textless{}{-}} \FunctionTok{as.vector}\NormalTok{(warm.fluxes}\FloatTok{.2}\NormalTok{)}
\NormalTok{warm.flux.values }\OtherTok{\textless{}{-}}\NormalTok{ warm.flux.values[warm.flux.values }\SpecialCharTok{\textgreater{}}\DecValTok{0}\NormalTok{]}


\CommentTok{\# Basic statistics}
\FunctionTok{summary}\NormalTok{(warm.flux.values)}
\end{Highlighting}
\end{Shaded}

\begin{verbatim}
##      Min.   1st Qu.    Median      Mean   3rd Qu.      Max. 
## 1.600e-11 1.394e-09 4.548e-09 4.513e-08 1.370e-08 1.031e-05
\end{verbatim}

\begin{Shaded}
\begin{Highlighting}[]
\FunctionTok{summary}\NormalTok{(cold.flux.values)}
\end{Highlighting}
\end{Shaded}

\begin{verbatim}
##      Min.   1st Qu.    Median      Mean   3rd Qu.      Max. 
## 0.000e+00 1.470e-10 6.010e-10 3.590e-08 2.440e-09 1.409e-05
\end{verbatim}

\begin{Shaded}
\begin{Highlighting}[]
\CommentTok{\# Combine your data into one data frame}
\NormalTok{flux.data }\OtherTok{\textless{}{-}} \FunctionTok{data.frame}\NormalTok{(}
  \AttributeTok{value =} \FunctionTok{c}\NormalTok{(cold.flux.values, warm.flux.values),}
  \AttributeTok{temperature =} \FunctionTok{factor}\NormalTok{(}\FunctionTok{rep}\NormalTok{(}\FunctionTok{c}\NormalTok{(}\StringTok{"Cold (4°C)"}\NormalTok{, }\StringTok{"Warm (20°C)"}\NormalTok{), }
                           \AttributeTok{times =} \FunctionTok{c}\NormalTok{(}\FunctionTok{length}\NormalTok{(cold.flux.values), }\FunctionTok{length}\NormalTok{(warm.flux.values))))}
\NormalTok{)}

\CommentTok{\# Plot histogram, another way to visualize the data}
\FunctionTok{ggplot}\NormalTok{(flux.data, }\FunctionTok{aes}\NormalTok{(}\AttributeTok{x =}\NormalTok{ value, }\AttributeTok{fill =}\NormalTok{ temperature)) }\SpecialCharTok{+}
  \FunctionTok{geom\_histogram}\NormalTok{(}\AttributeTok{position =} \StringTok{"identity"}\NormalTok{, }\AttributeTok{bins =} \DecValTok{50}\NormalTok{, }\AttributeTok{alpha =} \FloatTok{0.6}\NormalTok{) }\SpecialCharTok{+}
  \FunctionTok{scale\_x\_log10}\NormalTok{() }\SpecialCharTok{+} 
  \FunctionTok{scale\_fill\_manual}\NormalTok{(}
    \AttributeTok{values =} \FunctionTok{c}\NormalTok{(}\StringTok{"Cold (4°C)"} \OtherTok{=} \StringTok{"\#1f77b4"}\NormalTok{,  }
               \StringTok{"Warm (20°C)"} \OtherTok{=} \StringTok{"\#ff7f0e"}\NormalTok{)}
\NormalTok{  ) }\SpecialCharTok{+}
  \FunctionTok{labs}\NormalTok{(}
    \AttributeTok{x =} \StringTok{"Log Energy flux (J/s)"}\NormalTok{,}
    \AttributeTok{y =} \StringTok{"Count"}\NormalTok{,}
    \AttributeTok{fill =} \StringTok{"Temperature"}
\NormalTok{  ) }\SpecialCharTok{+}
  \FunctionTok{theme\_light}\NormalTok{() }\SpecialCharTok{+}
  \FunctionTok{theme}\NormalTok{(}\AttributeTok{text =} \FunctionTok{element\_text}\NormalTok{(}\AttributeTok{size =} \DecValTok{16}\NormalTok{))}
\end{Highlighting}
\end{Shaded}

\includegraphics{flux_web_vignette_files/figure-latex/unnamed-chunk-9-1.pdf}

\begin{Shaded}
\begin{Highlighting}[]
\CommentTok{\# Total energy flux in each temperature scenario}
\NormalTok{total.warm.flux }\OtherTok{\textless{}{-}} \FunctionTok{sum}\NormalTok{(warm.flux.values, }\AttributeTok{na.rm =} \ConstantTok{TRUE}\NormalTok{)}
\NormalTok{total.cold.flux }\OtherTok{\textless{}{-}} \FunctionTok{sum}\NormalTok{(cold.flux.values, }\AttributeTok{na.rm =} \ConstantTok{TRUE}\NormalTok{)}

\CommentTok{\# Create a data frame for plotting}
\NormalTok{total.flux.df }\OtherTok{\textless{}{-}} \FunctionTok{data.frame}\NormalTok{(}
  \AttributeTok{temperature =} \FunctionTok{c}\NormalTok{(}\StringTok{"Cold (4°C)"}\NormalTok{, }\StringTok{"Warm (20°C)"}\NormalTok{),}
  \AttributeTok{total\_flux =} \FunctionTok{c}\NormalTok{(total.cold.flux, total.warm.flux)}
\NormalTok{)}

\CommentTok{\# Bar plot}
\FunctionTok{ggplot}\NormalTok{(total.flux.df, }\FunctionTok{aes}\NormalTok{(}\AttributeTok{x =}\NormalTok{ temperature, }\AttributeTok{y =}\NormalTok{ total\_flux, }\AttributeTok{fill =}\NormalTok{ temperature)) }\SpecialCharTok{+}
  \FunctionTok{geom\_col}\NormalTok{(}\AttributeTok{width =} \FloatTok{0.6}\NormalTok{) }\SpecialCharTok{+}
  \FunctionTok{scale\_fill\_manual}\NormalTok{(}\AttributeTok{values =} \FunctionTok{c}\NormalTok{(}\StringTok{"Cold (4°C)"} \OtherTok{=} \StringTok{"\#1f77b4"}\NormalTok{, }\StringTok{"Warm (20°C)"} \OtherTok{=} \StringTok{"\#ff7f0e"}\NormalTok{)) }\SpecialCharTok{+}
  \FunctionTok{labs}\NormalTok{(}
    \AttributeTok{title =} \StringTok{"Total Energy Flux in Food Web"}\NormalTok{,}
    \AttributeTok{y =} \StringTok{"Total Flux (J/s)"}\NormalTok{,}
    \AttributeTok{x =} \StringTok{"Temperature"}\NormalTok{,}
    \AttributeTok{fill =} \StringTok{""}
\NormalTok{  ) }\SpecialCharTok{+}
  \FunctionTok{theme\_light}\NormalTok{()}
\end{Highlighting}
\end{Shaded}

\includegraphics{flux_web_vignette_files/figure-latex/unnamed-chunk-9-2.pdf}

The results from this scenario align with expectation for real
communities. Warming is likely to filter out certain species-
particularly those adapted to cold environments or occupying higher
trophic levels (e.g., generally those with low biomasses). This
filtering effects reduces the magnitude of the energy flux distribution
due to the loss of organisms from the food web. As a result, the total
energy demand decreases and becomes lower than in the cold scenario,
even though energetic losses to metabolism are higher under warming.
This suggests that warming can shift the energetic budget of food webs.
Moreover, when species are lost, structural changes may also occur if
remaining species begin exploiting new food resources, potentially
reconfiguring trophic interactions.

\end{document}
